\section{Umsetzung}
Wie bereits in \ref{Neues_Testkonzept} definiert wird der Tester mit einer Konfigdatei und mehreren Pythondateien umgesetzt. \todo{sicherstellen, dass das in "Neues Testkonzept" steht} Zusätzlich werden die Realen Gipfel für jeden Testfall in einer seperaten Json-Datei definiert und die Ergebnisse in selbigen Format gespeichert. 

Der Tester besteht aus zwei Klassen. Sie heißen Test\_Runner und Test\_Controler. 

\todo{Diagramm uml of test\_controler}

Der Test\_Controler ist für die Haushaltung und initialisierung der Tests zuständig. 
\todo{beschreibung definition der Funktion des Test\_Controler}
Er führt außerdem die Tests der Laufzeitdauer durch, da diese wenig Programmcode zur Ausführung benötigen und eine andere Ausführlogik sowie Metriken wie die Genauigkeitstests besitzen.

\todo{Diagramm uml of test\_runner}

Der Test\_Runner ist für die Ausführung der Genauigkeitstests zuständig. 
\todo{bescchreibe Test\_runner}
