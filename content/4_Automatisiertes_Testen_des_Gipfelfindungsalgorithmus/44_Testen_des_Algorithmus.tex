\section{Testen des Algorithmus}
In diesem Unterkapitel werden die Test-Szenarien auf die der Tester getestet wird beschreiben. Außerdem wird die Beschaffung der Testdaten erläutert und die Testergebnisse bewertet. 

\subsection{Definition der Test-Szenarien}
Um die korrekte Funktion des Algorithmus zu prüfen wird auf verschiedenen Kartenausschnitten getestet. Dazu wurden drei Kartenauschnitte ausgewählt, die verscheidenen Fälle abbilden in welchen der Algorithmus eingesetzt werden kann. 

\todo{bild Wetterstein}

Der erste Kartenausschnitt konzentriert sich auf das Wettersteingebirge (Alpen) bei Gamisch-Partenkirchen und große Teile des Karwendelgebirges nördlich von Innsbruck (Österreich), sowie viele umliegende Berge. Dieser Kartenausschnitt wurde ausgewählt, da er viele Bergrücken und Grate beinhaltet, die in diverse Richtungen abfallen und denoch weitgehend freistehende Berge wie die Große Arnspitze oder den Vorderskopf beinhaltet. Dieser Kartenabschnitt bietet zudem vielfältige Berghöhen und gleichzeitig viele Gipfel, die Höhentechnisch nahe bei einander sind. \todo{umformulieren} 

\todo{bild Feldberg Kartenausschnitt}

Der zweite Kartenausschnitt zentriert sich um den höchsten Berg des Schwarzwaldes, den Feldberg. Dieser Bereich wurde gewählt, da der Algorithmus bis jetzt weniger in Mittelgebirgen getestet wurde. In Mittelgebirgen sind die Hänge meist weniger steil und die Gipfel weniger eindeutig. Die Whal fiel auf die Region um den Feldberg, da diese Region gut kartographiert ist und daher die Gipfel schon genau bestimmt sind. Außerdem gleicht der Feldberg auf seinem Gipfel mehr einer Hochebene als einem herkömlichen Gipfel. Dieses Terrain ist daher ein gutes Gegenstück zu den schroffen und unebenen nördlichen Kalkalpen.  

\todo{bild Süd Schwarzwald}

Der letzte Kartenausschnitt ist ein großer Kartenausschnitt des Süd Schwarzwalds. Die Wahl fällt auf den Südschwarzwald, da er sehr vielseitig ist. Er fällt in richtung Süden und Westen steil ins Rheintal ab und wandelt sich in den Osten in eine Hügellandschaft. Dieser Test soll, durch das einschließen des Rheintals bei Schaffhausen außerdem zeigen, wie sich der Algorithmus auf weniger gut karthographiertem Gebeit bedienen lässt und ob er Erhebungen finden, die nicht in den Datenquellen enthalten sind.

Es wurde kein Abschnitt in Schweizer Alpen gewählt, da der Algorithmus dort schon mehr getestet wurde.
\todo{Inhalt:
- Definition der konkreten Testszenarien 
- Grund für deren Wahl
- Daten herbekommen?
- Schwierigkeiten, die überprüft werden sollen
- Durchlauf der Tests}
