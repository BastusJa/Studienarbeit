\section{Testen des Algorithmus}
In diesem Unterkapitel werden die Test-Szenarien auf die der Tester getestet wird beschreiben. Außerdem wird die Beschaffung der Testdaten erläutert und die Testergebnisse bewertet. 

\subsection{Definition der Test-Szenarien}
Um die korrekte Funktion des Algorithmus zu prüfen wird auf verschiedenen Kartenausschnitten getestet. Dazu wurden drei Kartenauschnitte ausgewählt, die verscheidenen Fälle abbilden in welchen der Algorithmus eingesetzt werden kann. 

\todo{bild Wetterstein}

Der erste Kartenausschnitt konzentriert sich auf das Wettersteingebirge (Alpen) bei Gamisch-Partenkirchen und große Teile des Karwendelgebirges nördlich von Innsbruck (Österreich), sowie viele umliegende Berge. Dieser Kartenausschnitt wurde ausgewählt, da er viele Bergrücken und Grate beinhaltet, die in diverse Richtungen abfallen und denoch weitgehend freistehende Berge wie die Große Arnspitze oder den Vorderskopf beinhaltet. Dieser Kartenabschnitt bietet zudem vielfältige Berghöhen und gleichzeitig viele Gipfel, die Höhentechnisch nahe bei einander sind. \todo{umformulieren} 

\todo{bild Feldberg Kartenausschnitt}

Der zweite Kartenausschnitt zentriert sich um den höchsten Berg des Schwarzwaldes, den Feldberg. Dieser Bereich wurde gewählt, da der Algorithmus bis jetzt weniger in Mittelgebirgen getestet wurde. In Mittelgebirgen sind die Hänge meist weniger steil und die Gipfel weniger eindeutig. Die Whal fiel auf die Region um den Feldberg, da diese Region gut kartographiert ist und daher die Gipfel schon genau bestimmt sind. Außerdem gleicht der Feldberg auf seinem Gipfel mehr einer Hochebene als einem herkömlichen Gipfel. Dieses Terrain ist daher ein gutes Gegenstück zu den schroffen und unebenen nördlichen Kalkalpen.  

\todo{bild Südschwarzwald}

Der letzte Kartenausschnitt ist ein großer Kartenausschnitt des Südschwarzwalds. Die Wahl fällt auf den Südschwarzwald, da er sehr vielseitig ist. Er fällt in richtung Süden und Westen steil ins Rheintal ab und wandelt sich in den Osten in eine Hügellandschaft. Dieser Test soll, durch das einschließen des Rheintals bei Schaffhausen außerdem zeigen, wie sich der Algorithmus auf Gebiet bedienen lässt, wo nicht alle Gipfel in der Datenquelle zwingend bekannt sind.

Es wurde kein Abschnitt in den Schweizer Alpen gewählt, da der Algorithmus dort schon in \todo{zitat zu Tester-Studienarbeit} getestet wurde.

Die Vergleichsbereich werden bei den Tests jeweils jeweils einmal als \textit{$\sqrt{2}$ * Pixelgröße}, \textit{100m} und \textit{minimale Dominanz} gewählt. Das sorgt dafür, dass schnell erkennbar ist welche Gipfel auf welcher Genauigkeit erkannt wurden. Dabei beschreibt \textit{$\sqrt{2}$ * Pixelgröße} die maximale Genauigkeit mit der eine pixelbassierte Karte einen Gipfel voraussagen kann. Die \textit{100m} sind der Bereich, den wir als gewünschte Genauigkeit des Algorithmus definiert haben. Die \textit{minimale Dominanz} beschreibt nach \ref{Neues_Testkonzept} den maximalen Bereich in dem wir die realen Gipfel den gefundenen zuordnen können.

Es wurde kein Abschnitt in den Schweizer Alpen gewählt, da der Algorithmus dort schon in \todo{zitat zu Tester-Studienarbeit} getestet wurde.

Die Vergleichsbereich werden bei den Tests jeweils jeweils einmal als \textit{$\sqrt{2}$ * Pixelgröße}, \textit{100m} und \textit{minimale Dominanz} gewählt. Das sorgt dafür, dass schnell erkennbar ist welche Gipfel auf welcher Genauigkeit erkannt wurden. Dabei beschreibt \textit{$\sqrt{2}$ * Pixelgröße} die maximale Genauigkeit mit der eine pixelbassierte Karte einen Gipfel voraussagen kann. Die \textit{100m} sind der Bereich, den wir als gewünschte Genauigkeit des Algorithmus definiert haben. Die \textit{minimale Dominanz} beschreibt nach \ref{Neues_Testkonzept} den maximalen Bereich in dem wir die realen Gipfel den gefundenen zuordnen können.

\subsection{Beschaffung der Testdaten}
Die Testdaten wurden aus verschiedenen Quellen beschafft. Die Geodaten (GeoTIFF-Dateien) wurden von verschiedenen Quellen beschafft. Für die Alpen wurden die Daten des Copernicus Programms der EU \todo{abkürzung} bezogen. Diese Entscheidung wurde Aufgrund der Ländergrenze zwischen Deutschland und Österreich, welche sich in dem Kartenausschnitt befindet, getätigt. Als Kartenmodell wurde hier die \enquote{Digital Elevation Model (DEM) for Europe at 30 meter} gewählt. Aus ihr können Kartenausschnitte von Europa im mit einer Auflösung von 30m pro Pixel heruntergeladen werden.

Für den Schwarzwald wurden die Daten von der Landesvermessung Baden-Württemberg bezogen. Der Grund hierfür ist, dass die Daten dort genauer sind. Alle Daten wurden in einer Auflösung von 0.25m pro Pixel bezogen. Da diese Auflösung zu hoch für die Testzwecke ist, wurden die Daten auf eine Auflösung von 10m pro Pixel heruntergerechnet.

Die Daten über die realen Gipfel wurden aus verschiedenen Quellen bezogen. Da in den Daten nicht flächendeckend  die Prominenz und Dominaz definiert wruden, mussten bei der Erstellung der Realdatensätze verschiedene API-Quellen und manuelle Suche kombiniert werden. Zum einen wurden die Daten von der OpenStreetMap API bezogen. Diese Daten wurden mit Hilfe der Overpass API abgefragt. Es wurden alle Punkte mit dem Tag \textit{natural=peak} innerhalb der Kartenausschnitte abgefragt. Diese Daten wurden wurden zunächst mit den Daten von Wikidata abgeglichen und durch diese ergänzt. Wikidata bietet eine API an, über die für bestimmte Koordinaten die Gipfel abgefragt werden können. Mit diesem Abgleich konnten über mehrere Kartenbereichen in den Alpen im Durchschitt 80\% der Gipfel mit Prominenz über 100m gefunden werden. Dies leigt nicht daran, dass die Gipfel selbst unbekannt sind, sondern daran, dass die Prominenz und Dominanz nicht in den Datenquellen definiert wurden. Daher wurden die Daten von Hand ergänzt und auf Vollständigkeit überprüft. Zum besseren Analyseren der Tests wurden die Gipfel, die nicht den Kriterien für einen Gipfel entsprechen, als \textit{Fake Peaks} definiert und in die Testdaten aufgenommen. Es wurden Punkte ausgewählt, die in der Nähe von realen Gipfeln liegen, aber nicht die Kriterien für einen Gipfel erfüllen oder von den APIs als Gipfel, die den Bedingungen nicht entprechen erkannt wurden. Diese Daten helfen bei dem schnellen finden von Fehlern des Algorithmus, da sie zeigen, welchen Gipfel der Algorithmus fälschlicherweise als Gipfel identifiziert hat und nicht nur seine Koordinaten.
Um die Daten manuell zu ergänzen und zu überprüfen muss die Karte von Hand durchsucht werden. Dabei stellt sich die Frage, welche Karte verwendet werden sollte, um die genausten testdaten zu erhalten. Dabei spielen Faktoren wie die allgemeine Genauigkeit der Höhenlinien und Anzahl der makierten Gipfel eine Rolle. Die Höhenlinien sind wichtig um die Prominenz abzuschätzen, wenn keine Realwerte dafür existieren. Bei wie vielen Gipfel keine Realwerte in den Datenquellen existieren hängt von vielen Faktoren ab. Dabei gilt meist, dass desto berühmter, höher die Gipfel und desto höher deren Prominenz und Dominanz ist umso wahrscheinlicher existieren Realwerte für diese Berge und umso besser sind diese. Ersteres spielt nach der Erfahrung eine größere Rolle. So ist beispielsweise die Höhe der Zugspitze in der Wikidata Datenbank auf Centimeter genau definiert, während die Höhe der meisten anderen Gipfel nur auf Meter genau definiert ist. Es gibt aber auch Fälle, wie beispielsweise den Daniel, bei der die Höhe nur auf 10 Meter genau definiert ist, obwohl sie ein eine große Prominenz von über 500m besitzt.

Es wurden verschiedene Karten verglichen, wie die OpenStreetMap Karte, die Karte von Google Maps und die Karte von Bing Maps. Es wurde festgestellt, dass die OpenStreetMap Karte die genausten Höhenlinien und die meisten markierten Gipfel enthält. Daher wurde sie als Hauptquelle für die manuelle Ergänzung der Testdaten verwendet. Die Werte für die Prominenz und Dominanz wurden dabei durch Abschätzen der Höhenlinien auf der Karte ergänzt, wenn nicht schon durch die APIs gegeben.

\subsection{Durchlauf der Tests}
Nachdem die Testdaten erstellt wurden, wurden die Tests mit verschiedenen Konfigurationen des Algorithmus durchgeführt. Dabei wurden verschiedene Werte für die Schwellenwerte der Prominenz und Dominanz getestet, um zu sehen, wie sich diese auf die Genauigkeit der Gipfelerkennung auswirken. Dabei ist aufgefallen, dass die Datenqualität aprupt abnimmt, wenn der Schwellenwert für die Prominenz und der Dominanz zu neidrig angesetzt werden. Um einerseits den Gipfelfinder in schwerem Gebiet zu Testen und gleichzeitig die Datenqualität zu erhalten, wurde bei keinem Test der Schwellenwert für die Prominenz unter 100m und der Schwellenwert für die Dominanz unter 1000m gesetzt. Es wurden aber auch Tests mit höheren Schwellenwerten durchgeführt, um zu sehen, wie sich diese auf die Genauigkeit auswirken.


\subsection{Bewertung der Testergebnisse}
Nach dem durchlaufen der Tests wurden die Ergebnisse ananlysiert und bewertet. Dabei wurde nicht nur die Genauigkeit der Gipfelerkennung bewertet, sondern auch die durchschnittlichen und extremen Abweichungen für Position, Höhe, Prominenz und Dominanz, sowie die Laufzeit. Es wrude festgestellt, dass der Algorithmus bei den meisten Tests die Position der Gipfel mit einer hohen Genauigkeit bestimmt. Die Höhe der Gipfel wird ebenfalls mit einer hohen Genauigkeit bestimmt, wobei die durchschnittliche Abweichung meist unter 10 Metern liegt. Diese Abweichungen lassen sich durch die Auflösung und Qualität der Geodaten, sowie Realdaten erklären. 

Es gibt Fälle in welchen der Algorithmus Gipfel erkennt, die in Realität nicht existieren. Das passiert nur in der Nähe des Kartenrandes. Das liegt daran, dass kleine Erhebungen am Hang eines Berges nahe des Kartenrandes nicht als Gipfel ausgeschlossen  werden können. In diesem Fall wird ihnen der Hang des gesammten Berges bei der Prominenzberechnung angerechent. \todo{diagram mit Zylinder oder berg und dann ne border durch/ höhenlinien an border} Um dies zu verhindern wurde bereits die \textit{border width} eingeführt. Diese scheint aber nicht richtig zu funktionieren, die Randwertfehler nun am Rand der \textit{border width} auftretten.

\todo{das border width problem diagramm}

Die Prominenz und Dominanz werden ebenfalls mit einer guten Genauigkeit bestimmt, wobei die durchschnittliche Abweichung meist unter 20 Metern liegt. Die Laufzeit des Algorithmus variiert je nach Größe des Kartenausschnitts und der Anzahl der gefundenen Gipfel, liegt aber in der Regel im Bereich von einigen Sekunden bis zu einigen Minuten. \todo{das stimmt nicht, werte ändern}

\todo{birkkarspitzenproblem, wannig hat das auch}

\todo{In dem alten Algorithmus wurde angenommen, dass die Prominece immer zum nächsten Gipfel geht. Das muss sie nicht. 
Das spiegelt der neue Algorithmus mit einer modifierten Abbruchbedingung und keinem Endpunkt wieder.}

\todo{analyse auf altem project}

\todo{abschlißend die schwierigkeiten des algorithmus als Topografische bereiche}

\todo{Inhalt:
- Definition der konkreten Testszenarien 
- Grund für deren Wahl
- Daten herbekommen?
- Schwierigkeiten, die überprüft werden sollen
- Durchlauf der Tests}
